\lstset{
language=xml,
tabsize=3,
frame=lines,
label=code:xml,
%frame=framebox,
rulesepcolor=\color{gray},
xleftmargin=20pt,
framexleftmargin=15pt,
keywordstyle=\color{blue}\bf,
commentstyle=\color{OliveGreen},
stringstyle=\color{red},
numbers=left,
numberstyle=\tiny,
numbersep=5pt,
breaklines=true,
showstringspaces=false,
basicstyle=\footnotesize,
caption=XML file structure}

\section{Input}
\label{sec:input}
\subsection{XML}
\label{subsection:xml}
The input XML file \texttt{parameters.xml} contains the parameter and corresponding value for the functions.
Additionally, it specifies the information of the aircraft that those correspond to.
The basic syntax of an XML file is as follows, \texttt{\textless parameter \textgreater value \textless /parameter\textgreater}; for example \texttt{\textless n\_ult \textgreater 4.5 \textless/n\_ult\textgreater} specifies the ultimate load factor parameter with a value of 4.5.

Since the input data for three methods are in the XML file, there needs to be a structure.
The following structure is used for the XML file,
\lstinputlisting{code/structure.xml}
For multiple methods in one file, the method structure part is simply copied.

Some parameters are common in different methods, for example the ultimate load factor is present in all three methods.
These common parameters are specified for each method, so the ultimate load factor is specified three times.
The reason for this is to have more flexibility, if the user wants to test a different value for one method and keep the others the same this will be possible.
