\definecolor{Code}{rgb}{0,0,0}
\definecolor{Decorators}{rgb}{0.5,0.5,0.5}
\definecolor{Numbers}{rgb}{0.5,0,0}
\definecolor{MatchingBrackets}{rgb}{0.25,0.5,0.5}
\definecolor{Keywords}{rgb}{0,0,1}
\definecolor{self}{rgb}{0,0,0}
\definecolor{Strings}{rgb}{0,0.63,0}
\definecolor{Comments}{rgb}{0,0.63,1}
\definecolor{Backquotes}{rgb}{0,0,0}
\definecolor{Classname}{rgb}{0,0,0}
\definecolor{FunctionName}{rgb}{0,0,0}
\definecolor{Operators}{rgb}{0,0,0}
\definecolor{Background}{rgb}{1,1,1}

\lstnewenvironment{python}[1][]{
\lstset{
numbers=left,
numberstyle=\footnotesize,
numbersep=1em,
xleftmargin=1em,
framextopmargin=2em,
framexbottommargin=2em,
showspaces=false,
showtabs=false,
showstringspaces=false,
frame=l,
tabsize=4,
% Basic
basicstyle=\ttfamily\small\setstretch{1},
backgroundcolor=\color{Background},
language=Python,
% Comments
commentstyle=\color{Comments}\slshape,
% Strings
stringstyle=\color{Strings},
morecomment=[s][\color{Strings}]{"""}{"""},
morecomment=[s][\color{Strings}]{'''}{'''},
% keywords
morekeywords={import,from,class,def,for,while,if,is,in,elif,else,not,and,or,print,break,continue,return,True,False,None,access,as,,del,except,exec,finally,global,import,lambda,pass,print,raise,try,assert},
keywordstyle={\color{Keywords}\bfseries},
% additional keywords
morekeywords={[2]@invariant},
keywordstyle={[2]\color{Decorators}\slshape},
emph={self},
emphstyle={\color{self}\slshape},
%
caption=Torenbeek example function,
label=fig:methods
}}{}


\subsection{Methods}
\label{subsec:methods}
In \texttt{methods.py} functions for the three methods are specified, these functions are from literature (see references).
A class for each method provides a structure for the location of each function.
Functions can have arguments that set a coefficient to a specific value depending on the value of the argument.
For example in the Torenbeek functions there is the option to specify the unit system, imperial or SI.
Using an if-statement the unit system can be easily changed, see Listing~\ref{fig:methods}.
\clearpage
\begin{python}
def w_w(self, w_g, b_ref, Lambda, b, n_ult, s, t_r, unit):
    if unit == 'im':
            k_w = 0.0017
        elif unit == 'si':
            k_w = 0.00667
        else:
            print "USAGE: 'unit' is 'im' or 'si'."

        return k_w*w_g*(b_ref/cos(radians(Lambda)))**0.75*(1 + (b_ref*cos(radians(Lambda))/b)**0.5)*n_ult**0.55*(b*s/(w_g*t_r*cos(radians(Lambda))))**0.3
\end{python}

