\section{Computation}
\label{sec:comp}

\subsection{Black box}
\label{subsec:blackbox}
The computation is done in the black box \texttt{blackbox.py}, which makes use of the input data and methods.
First the parser (see Section~\ref{subsec:parser}) is imported and then  called, which returns four dictionaries, with three of them containing data for the methods and one containing the aircraft information.
To use the functions described in the \texttt{methods.py} file it is first imported.
The arguments of the functions are set by specifying the key of the desired element, this is done for all of the functions.
The results of those functions are stored in dictionaries, one for each method, and are printed in the terminal/console window.
Additionally, the pie charts and output XML file are created using the dictionaries.
The pie charts are saved automatically as \texttt{pie\_tor.png}, \texttt{pie\_ray.png} and \texttt{pie\_gd.png} for Torenbeek, Raymer and General Dynamics, respectively.
The output XML file is saved as \texttt{output.xml.}



