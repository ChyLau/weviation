\section{Introduction}
For document describes the code and how it is build up.
The code can be found in the delivered files and is also available online at \url{https://github.com/ChyLau/weviation/tree/master/weviation}.
Additionally, the logbook can be found at \url{https://github.com/ChyLau/weviation/commits/master} and gives a good indication of how much time is spend on this case study.

In Figure~\ref{fig:flow_bb} and~\ref{fig:flow_gui} the flow diagrams can be seen. Each diagram has three main parts: input, computation and output.
The input section is the same for both the black box and GUI, but they differ in the output part.
The black box can be used as a module for another project, and the GUI is for users.
For more customizability the GUI is used, which allows user interaction.

The following sections describe the parts of the flow diagram, starting with the Input (section~\ref{sec:input}), following with the Computation (section~\ref{sec:comp}) and ending with the Output (section~\ref{sec:output}).

Next the validation of the blackbox/GUI is described in section~\ref{sec:validation}, which check if each method gives a similar result.

\begin{figure}[h]
\centering
\begin{tikzpicture}[scale=0.7,transform shape]

  % Draw diagram elements
\path \practica{1}{Black box}{Computation of the methods};

\path (p1.north)+(-2.5,2.0) \practica{2}{Parser}{Extracts the parameters and values};
\path (p2.north)+(0.0,1.0) \practica{3}{XML}{Containing input data};

\path (p1.north)+(2.5,2) \practica{4}{Methods}{Torenbeek, Raymer, General Dynamics};

\path (p1.south)+(0.0,-2.0) \practica{5}{Pie chart}{Visual result of the estimation};
\path (p1.south)+(-5,-2.0) \practica{6}{XML}{Containing output data};
\path (p1.south)+(5,-1.9) \practica{7}{Terminal}{Printed result};

\path [line] (p1.south) -- node [above] {} (p5);
\path [line] (p1.south) -- +(0.0,-0.9) -- +(-5.0,-0.9) -- node [above, midway] {} (p6);
\path [line] (p1.south) -- +(0.0,-0.9) -- +(5.0,-0.9) -- node [above, midway] {} (p7);

\path [line] (p3.south) -- node [above] {} (p2);
\path [line] (p2.south) -- +(0.0, -0.9) -- +(2.5, -0.9) -- node [above, midway] {} (p1);
\path [line] (p4.south) -- +(0.0, -0.9) -- +(-2.5, -0.9) -- node [above, midway] {} (p1);
  % Draw arrows between elements
  %\path [line] (p1.south) -- node [above] {} (p2);

  %\path [line] (p2.south) -- +(0.0,-0.5) -- +(-2.5,-0.5)
  %  -- node [above, midway] {} (p3);
  %\path [line] (p3.south) -- node [above] {} (p5) ;

  %\path [line] (p2.south) -- +(0.0,-0.5) -- +(+2.5,-0.5)
  %  -- node [above, midway] {} (p4);
  %\path [linepart] (p3.east) -- +(+0.5,-0.0) -- +(+0.5,-1.75)
  %  -- node [left, midway] {} (p4);
  %\path [linepart] (p3.east) -- +(+0.5,-0.0) -- +(+0.5,-1.75)
  %  -- node [left, midway] {} (p4);

  %\path [line] (p4.south) -- +(0.0,-0.5) -- +(-2.5,-0.5)
  %  -- node [above, midway] {} (p6);
  %\path [line] (p5.south) -- +(0.0,-0.5) -- +(+2.5,-0.5)
  %  -- node [above, midway] {} (p6);
  %\path [linepart] (p2.east) -- +(2.75,0.0) -- +(2.75,-5.85)
  %  -- node [right] {} (p6);
  %\path [line] (p6.south) -- +(0.0,-0.25) -- +(-2.5,-0.25)
  %  -- node [above, midway] {} (p7);
  %\path [line] (p6.south) -- +(0.0,-0.25) -- +(+2.5,-0.25)
  %  -- node [above, midway] {} (p8);
  %\path [linepart] (p7.east) -- node [left] {} (p8);
  %\transreceptor{p8}{AM banda 40m}{ur1}

  %\path [line] (p7.south) -- node [above] {} (p9) ;
  %\path [line] (p8.south) -- node [above] {} (p10) ;
  %\path [linepart] (p9.east) -- node [left] {} (p10);
  %\transreceptor{p10}{CW}{ur2}
  %\path [line] (p9.south) -- node [above] {} (p11) ;
  %\path [line] (p10.south) -- node [above] {} (p12) ;
  %\path [linepart] (p11.east) -- node [left] {} (p12);
  %\transreceptor{p12}{FDMDV}{ur3}

  %\path [line] (p11.south) -- node [above] {} (p13) ;
  %\path [line] (p12.south) -- node [above] {} (p14) ;
  %\path [linepart] (p13.east) -- node [left] {} (p14);
  %\transreceptor{p14}{SSTV}{ur4}

  %\path [line] (p14.south) -- +(0.0,-0.5) -- +(-2.5,-0.5)
  %  -- node [above, midway] {} (p15);
  %\path [line] (p13.south) -- +(0.0,-0.5) -- +(+2.5,-0.5)
  %  -- node [above, midway] {} (p15);
  %\path [line] (p15.south) -- node [above] {} (p16) ;
  %\path [line] (p16.south) -- node [above] {} (p17) ;
  %\path [line] (p17.south) -- node [above] {} (p18) ;

  \background{p6}{p3}{p7}{p4}{Input}
  \background{p6}{p1}{p7}{p1}{Computation}
  \background{p6}{p5}{p7}{p5}{Output}
  %\background{p3}{p3}{p4}{p5}{II}
  %\background{p3}{p6}{p4}{p7}{III}
  %\background{p3}{p9}{p4}{p10}{IV}
  %\background{p3}{p11}{p4}{p12}{V}
  %\background{p3}{p13}{p4}{p14}{VI}
  %\background{p3}{p15}{p4}{p16}{VII}
  %\background{p3}{p17}{p4}{p17}{VIII}
  %\background{p3}{p18}{p4}{p18}{IX}
\end{tikzpicture}

\caption{Flow diagram for the black box.} \label{fig:flow_bb}
\end{figure}

\begin{figure}[h]
\centering
\input{image/diagram_gui.tex}
\caption{Flow diagram for the GUI.} \label{fig:flow_gui}
\end{figure}
